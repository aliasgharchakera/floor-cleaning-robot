\documentclass[12pt]{article}
\usepackage{graphicx}
\usepackage{geometry}
\usepackage{fancyhdr}
\usepackage{titletoc}
\usepackage{titlesec}
\usepackage{listings}

% Page setup
\geometry{
    top=1in,
    bottom=1in,
    left=1in, % Adjust the left margin
    right=1in, % Adjust the right margin
}

% breaklines=true,

\pagestyle{fancy}
\fancyhf{} % Clear all header and footer fields
\fancyhead[R]{Optimizing Cleaning Paths for Robots in Domestic Settings} % Add this line to set the header on the right side
\renewcommand{\headrulewidth}{0pt} % Remove the horizontal line in the header

% Define section numbering format
\titleformat{\section}{\normalfont\large\bfseries}{\thesection}{1em}{}
\titleformat{\subsection}{\normalfont\normalsize\bfseries}{\thesubsection}{1em}{}

% Title Page
\begin{document}
\begin{titlepage}
    \centering
    {\LARGE\textbf{EE/CE 468 : Mobile Robotics}\par}
    \vspace{0.5cm}
    {\Large Project Functional Requirements\par}
    \vspace{0.2cm}
    {\Large Optimizing Cleaning Paths for Robots in Domestic Settings\par}
    \vspace*{\fill} % Vertically center the logo and text
    {\large Ali Asghar Yousuf $\mid$ ay06993@st.habib.edu.pk\par}
    {\large Muhammad Azeem Haider $\mid$ mh06858@st.habib.edu.pk\par}
    \vspace{2cm}
    \includegraphics[height=7cm]{../HU_logo}\\\bigskip
    {\large \today}\\\bigskip\bigskip
    \vspace{1cm}
    \vspace{2cm}
    {\large Dhanani School of Science and Engineering\par}
    {\large Habib University\par}
    {\large Fall 2023\par}
    \vspace*{\fill} % Vertically center the copyright text
    {\large Copyright @ 2023 Habib University\par}
\end{titlepage}

% Index page
\thispagestyle{empty} % No page number on index page
\tableofcontents
\clearpage

\section{Introduction}
Cleaning paths for robots at homes are usually obstructed by furniture present
which leads to inefficient work. Mobile robots for cleaning at home such as the
``Roomba'' by iRobot, has been described as drunk due to them getting stuck
under furniture and stop working after hitting obstacles in their path. The
project aims to research the cleaning and obstacle finding algorithms of these
robots in domestic settings and study on how these can be implemented in a
manner where the path planning of the robot is much more efficient for cleaning
purposes.

\end{document}
