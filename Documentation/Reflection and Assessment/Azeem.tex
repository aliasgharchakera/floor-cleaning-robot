\documentclass[12pt]{article}
\usepackage{graphicx}
\usepackage{geometry}
\usepackage{fancyhdr}
\usepackage{titletoc}
\usepackage{titlesec}
\usepackage{listings}

% Page setup
\geometry{
  top=1in,
  bottom=1in,
  left=1in, % Adjust the left margin
  right=1in, % Adjust the right margin
}

% breaklines=true,

\pagestyle{fancy}
\fancyhf{} % Clear all header and footer fields
\fancyhead[R]{Reflection and Assessment Plan} % Add this line to set the header on the right side
\renewcommand{\headrulewidth}{0pt} % Remove the horizontal line in the header

% Define section numbering format
\titleformat{\section}{\normalfont\large\bfseries}{\thesection}{1em}{}
\titleformat{\subsection}{\normalfont\normalsize\bfseries}{\thesubsection}{1em}{}

% Title Page
\begin{document}
\begin{titlepage}
    \centering
    {\LARGE\textbf{EE/CE 468: Mobile Robotics}\par}
    \vspace{0.5cm}
    {\Large Reflection and Assessment Plan\par}
    \vspace*{\fill} % Vertically center the logo and text
    {\large Muhammad Azeem Haider $\mid$ mh06858@st.habib.edu.pk\par}
    \vspace{2cm}
    \includegraphics[height=7cm]{../HU_logo.png}\\\bigskip
    {\large \today}\\\bigskip\bigskip
    \vspace{2cm}
    {\large Dhanani School of Science and Engineering\par}
    {\large Habib University\par}
    {\large Fall 2023\par}
    \vspace*{\fill} % Vertically center the copyright text
    {\large Copyright @ 2023 Habib University\par}
\end{titlepage}

% Index page
\thispagestyle{empty} % No page number on the index page
% \tableofcontents
\clearpage

\section{Understanding Fundamental Concepts of Mobile Robotics}

Since the commencement of our course in August, we've delved into fundamental concepts in Mobile Robotics. However, I still find some of these concepts, such as frames of reference and the application of Rotation Matrix for switching frames, a bit perplexing. This confusion was notably reflected in my Quiz 1 grades. While self-reflection on abstract concepts like frames of reference might pose a challenge, I plan to revisit the slides or corresponding chapters in the textbook to evaluate my current understanding.

Throughout this project, my primary learning goal is to thoroughly comprehend these fundamental mobile robotics concepts to the best of my abilities. I aim to achieve this understanding through practical, hands-on work within the project's scope. This comprehension holds pivotal significance for me to deem this course a success. I plan to assess my progress by the end of the project through practical implementation and self-reflection on these concepts. Gaining a robust grasp of these principles is not only a goal for this course but also a foundational element for potential future personal projects in robotics. 

\subsection{Assessment Plan}

To assess my understanding of fundamental Mobile Robotics concepts, I will employ the following criteria:

\begin{enumerate}
    \item \textbf{Practical Implementation:} I will undertake specific tasks within the project that require the application of concepts taught in the course. Success in implementing these tasks will indicate a deep understanding. The success of these implementations can be gauged by the grades received.
    
    \item \textbf{Self-Reflection:} Regular self-reflection on my comprehension of fundamental concepts will be documented. I will write journal notes about it tri-weekly. This reflection will include challenges faced, solutions found, and areas that need further exploration.

    \item \textbf{Project Outcome Evaluation:} The success of the overall project, especially in areas directly linked to fundamental Mobile Robotics concepts, will serve as a comprehensive assessment of my understanding. Similar to \textit{Practical Implementation}, the outcomes/grades achieved on the project will also be a good self-check for evaluating if I understand fundamental concepts or not.  
\end{enumerate}

\section{Improving Matlab and Robotics Programming Skills}

As a student who transitioned from an Electrical Engineering major to a Computer Science major, I've had some exposure to using Matlab in courses like Electric Circuit Analysis and Mechanics and Thermodynamics labs. I believe that Matlab is an excellent platform, and I aspire to master it to a level where I can use it with the same ease as programming languages like Python. In the realm of robotics programming, my goal is to become proficient in coding mobile robots that operate efficiently and can successfully execute the functions our group plans to implement. By enhancing my robotics programming skills, I can even consider taking on more challenging projects, such as designing a mobile robot for deep-sea exploration in extremely harsh environments.

\subsection{Assessment Plan}

The following assessment plan will be implemented to assess my programming and Matlab skills:

\begin{enumerate}
    \item \textbf{Peer Evaluation:} Seeking help from ECE students who have taken the Introduction to Robotics course and are more fluent with Matlab and Robotics programming. Asking them for reviews and feedback will be a good way to assess how successful I have been in fulfilling this goal. 

    \item \textbf{Problem-Solving Capability:} The ability to troubleshoot and debug code effectively will be evaluated. I will document instances where I encountered challenges and how efficiently I resolved them. This documentation will help me get a better reflection on how efficient I am at solving problems or when I encounter difficulties.

    \item \textbf{Coding Proficiency:} I will assess my comfort level and efficiency in writing Matlab code for robotics applications. Success will be measured by the clarity, conciseness, and functionality of the code. The instructor will provide comments on the code files provided, and these comments will help me assess how proficient I really am. 
\end{enumerate}

\section{Path-Finding Algorithms}

The Path-Finding problem is of great interest to me, particularly in how it pertains to the way robots determine the optimal path for their movements. By the culmination of this project, my goal is to attain a high level of proficiency in Path-Finding algorithms, enabling me to successfully apply these algorithms to a variety of robots in order to identify the most efficient routes. This knowledge will prove invaluable not only in this project but also for any personal endeavors I might undertake in the future. For instance, the prospect of constructing a mobile robot for deep-sea exploration interests me. With a thorough grasp of Path-Finding algorithms, I would be well-equipped to embark on such an ambitious venture. By the end of our project and the result of that project, I will be able to gauge if I know enough about the path-finding problem and algorithms related to it. 

\subsection{Assessment Plan}

\begin{enumerate}
    \item \textbf{Algorithm Implementation:} The successful implementation of the algorithm would mean that the project is successful to a higher degree. A better algorithm implementation would result in a more efficient domestic cleaning robot.
    
    \item \textbf{Comparison to Results Found in Literature Review:} The results of the implementation of path-finding algorithms will be compared to implementations found in the literature review. This too will be a good method to understand how successful I was when implementing the Path-finding algorithms. This comparison will be a good metric to know if I know enough about the problem and whether I implemented it correctly. 
\end{enumerate}

The assessment plan depends on reviews from the instructor and peers, but what is also important is to keep a journal and document my progress so I could notice in writing how I improved and, more importantly, if I improved in my learning goals for the project. While my learning goals remain the same, the assessment of these learning goals will be of great importance moving forward. 

\end{document}